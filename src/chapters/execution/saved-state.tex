% !TEX root = ../../main.tex

\section{Saved state} % (fold)
\label{sec:saved_state}

The \acrshort{api} also allows for an initialisation with a certain state passed in. This means that it's possible to save a state Object in a location, and then later restart the search from that data. 

This is useful for three major use cases. The first is \acrshort{url} synchronisation. For \acrshort{url} synchronisation the structure would be to first listen on every state change with {\tt .subscribe()}, and then asynchronously transform that to query parameters to set. If a user comes to a page with query parameters, a function to take in the query string and put out a state Object would then be called. After that point a new InstantSearch Core instance can be built with the {\tt preloadedState} parameter set in the constructor.

\begin{figure}[H]
  \centering
  \includegraphics[width=0.6\textwidth, height=0.2\textheight, draft]{../assets/is-core-url-sync.pdf}
  \caption{Synchronisation with the querystring using InstantSearch Core}
  \label{figure:is-core-url-sync}
\end{figure} %%

Similarly it is also possible to do that process with a different medium than query parameters, for example {\tt localStorage}. For that reason this process is left over to the user, possibly with other small packages dealing with browser inconsistencies on top of it. 

This pattern is also useful for when \acrfull{ssr} is implemented. When ...%%

\begin{enumerate}
  \item execute the IS function once
  \item render the output as html
  \item also output the state Object in a global Object
  \item send that html to the client
  \item start the IS instance frontend with preloadedState: {\tt window.PRELOADED\_STATE} %%
\end{enumerate}

\begin{figure}[H]
  \centering
  \includegraphics[width=0.6\textwidth, height=0.2\textheight, draft]{../assets/is-core-ssr.pdf}
  \caption{Server Side Rendering with InstantSearch Core}
  \label{figure:is-core-ssr}
\end{figure} %%

% section saved_state (end)
