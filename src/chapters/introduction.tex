% !TEX root = ../main.tex

\chapter{Introduction} % (fold)
\label{chp:introduction}

\textbf{Chapter \ref{chp:presentation}} describes the company Algolia, where the internship took place and where the research was done. First the history of the startup is briefly described. The importance of the company's culture for the founders and how this is translated in core values which determine the day-to-day operations are described in detail.

The research question and the operational goals of the research linked to the internship are explained in \textbf{chapter \ref{chp:research_question}}. 
In \textbf{chapter \ref{chp:action_plan}} the action plan of the internship and practical research are visualised.

Exploring the existing Algolia libraries and how they can be used in a large production website is an important part of this research as there are already several previous libraries to make search user interfaces with Algolia. The choice of the production website Yarn\cite{yarn-site} and how Algolia supports their clients are described in the first part of \textbf{chapter \ref{chp:exploratory_study}}. This is followed by an elaboration in the working of the consecutive libraries: Algolia JavaScript client, Algolia JavaScript Helper, InstantSearch.js, React InstantSearch. How the future looks like when choosing for InstantSearch Core closes the exploratory study.

\textbf{Chapter \ref{chp:execution}} contains the technical development done during this research. The chapter commences with the terminology needed when making an Algolia \acrshort{api} client. The main complexity is how to handle state within a \acrfull{rest} \acrshort{api} which is being used to make continuous interfaces. 

This chapter is divided by several complexities found when handling state. This is firstly how to organize state in an Algolia context. Continued by how to handle the updates of state either by the user or by new \acrshort{api} responses. This is followed by the challenges posed while rendering the responses, while finally some other use-cases are described.

In \textbf{chapter \ref{chp:conclusion}} the conclusion of the practical research during the internship is described. This is followed by a \textbf{bibliography} and an \textbf{appendix}.
