% !TEX root = ../main.tex

\chapter{Introduction} % (fold)
\label{chp:introduction}

This thesis is divided in two parts. The first part explores the existing Algolia libraries and how they can be used in a large production website and is described in chapter \ref{chp:exploratory_study}. The second part describes how to find a way to make the InstantSearch experience available for a multitude of frameworks and is described in chapter \ref{chp:execution}.

The place to discover the complete internals of InstantSearch and React InstantSearch that was chosen is the Yarn website~\cite{yarn-site}~. Yarn is a package manager built on top of npm~\cite{npm-github}~. It has been built from the ground up with a new registry, mirroring all changes to npm, and with a new client that only supports modern environments.

In the end of December, Algolia launched search on the Yarn website by replicating this registry again, but this time in an Algolia index. A search experience in InstantSearch.js was added~\cite{yarn-pr-add-algolia} at that time. To make this experience better --- after a meeting with core team members of the Yarn project --- the decision was made to improve how and where the search was, as well as making a detail page. This is described in the first the exploratory study, section \ref{sec:discovery}.

The current state of handling the state of a search using Algolia methods, as well as the history of this process is described in chapter \ref{sec:making_search_interfaces_with_algolia}.

The main goal is making {\tt instantsearch-core}. InstantSearch.js was released as a library to ease the use of working with Algolia in 2015, which was built on top of the helper~\cite{algolia-js-helper} directly, and called the methods to refine at helper-level. InstantSearch Core is a layer of abstraction to set state based on widgets and asks Algolia for the results of a certain ``widget configuration''. The result of this will allow for more rapid development of instantsearch for frameworks other than React~\cite{react-doc}~.

InstantSearch Core is based on the learnings when developing {\tt react-instantsearch}~\cite{react-instantsearch}~.  When that was developed, the need for another layer of abstraction was identified. Users don't need to know which exact facet or which helper method is being called, even when making custom {\tt connectors}\cite{react-instantsearch-connectors}~. While {\tt react-instantsearch} is still making use of the helper, a second layer~\cite{react-instantsearch-search-state} of state has been made. The process of making InstantSearch Core, and all its intricacies is described in chapter \ref{chp:execution}.
