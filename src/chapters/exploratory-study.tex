% !TEX root = ../main.tex

\chapter{Exploratory study} % (fold)
\label{chp:exploratory_study}

\section{Discovery}
\label{sec:discovery}

A way to discover all intricacies of an ecosystem is to use it in a lot of different ways. Before jumping into making a new library or improving an existing one, one should use it sufficiently.

To do that, two main methods are used. Firstly it is to convert an existing search implementation using InstantSearch.js\cite{instantsearch-js} to use React InstantSearch in a more complex heavily used production website. For that purpose the website of the package manager Yarn\cite{yarn-site} was chosen.

The other method chosen for this goal is being a full member of the support experience at Algolia. Every developer takes care of their own libraries in all support channels, which helps a lot in finding out whether certain things are wanted by customers, or certain others aren't. This also helps in finding bugs and understanding weaknesses in the product.

\subsection{Yarn} % (fold)
\label{ssec:yarn}

Yarn is an open source package manager developed jointly by engineers at Facebook, Google and Tilde for the JavaScript ecosystem, based on the npm package manager. It allows for code to be easily shared among everyone in the world~\cite{yarn-site}~.

In both the npm and Yarn ecosystem code is shared in packages --- also called modules --- which all have a {\tt package.json} file that describes metadata on the package, together with all the code. Yarn and npm share the npm repository for having the packages of the ecosystem.

The Algolia team already worked on integrating Algolia search into the Yarn website~\cite{yarn-pr-add-algolia}~. Since the very beginning of 2017, it’s possible to look up packages at a very impressive speed at \href{https://yarnpkg.com/search}{yarnpkg.com/search}. Making small fixes in the codebase\footnote{These fixes are the GitHub pull requests \href{https://github.com/yarnpkg/website/pulls/208}{yarnpkg/website\#208}, \href{https://github.com/yarnpkg/website/pulls/210}{yarnpkg/website\#210}, \href{https://github.com/yarnpkg/website/pulls211}{yarnpkg/website\#211} and \href{https://github.com/yarnpkg/website/pulls/335}{yarnpkg/website\#335} respectively} in the past made the project and research on further improving that search experience proved to be a very exciting endeavour.

The metadata in the npm and Yarn repository is being enriched by an indexing process\cite{npm-search} that takes in packages in the npm repository, and joins that together with places that have more information, like GitHub, and the amount of monthly downloads per package.

This search experience was such a success that one week prior to the start of the internship in Paris, some of the team members (Sylvain Utard, Kevin Granger and Vincent Voyer) went to the Facebook offices in London --- Facebook has the main contributors to the Yarn project --- to explore the possibilities of making a detail page for the search, and integrating it even more into the website.

The Yarn project is evolving from having metadata from the npm packages in an index, used only on the Yarn website, to a free to use API, which can make great search available to anyone who wants to make a project leveraging the ecosystem. The first other website that will use this data will be the online code editor \href{https://codesandbox.io}{CodeSandbox.io} for the interface where users can add dependencies to their projects. A lot of other integrations are expected in the time coming after this service will be formally announced.

% subsection yarn (end)
\subsection{Support}
\label{support}

At Algolia, customer support is not done by specific employees hired to do customer requests, but actually by the engineers and developers working on the code. This is a very varied list of people, who range from knowing a lot about a particular client or frontend library, since they wrote it, to people working on the SaaS client serving all requests.

This has as an effect that when someone starts in the company, after a few weeks, they get to interact with clients who use Algolia in very varied ways. Connecting to the users keeps the maintainers in an advantageous position, since it's very simple to find out what features to focus on, and bugs that weren't anticipated.

It also has a secondary effect, which is that it keeps everyone up to date with what is possible with a lot of different implementations.

This is also a venue in which in a short time, deep knowledge about the API parameters and the layers of abstraction is acquired. This has a similar influence on the understanding of libraries as using the libraries in a production website.

\section{Making search interfaces with Algolia} % (fold)
\label{sec:making_search_interfaces_with_algolia}

There have been several phases in the life of Algolia regarding building user interfaces on the web. In short, first there was only the REST API with the Algolia JavaScript client. Following to that, tutorials were written on how to use Algolia to build full search interfaces using technologies like jQuery, moment.js and other libraries. As a result keeping track of the search state is left over completely to the users to implement.

In the context of a search experience, the state of a page are the active refinements. This can often be seen in a specific widget, which is called {\tt currentRefinements} when using Algolia libraries. To make complex applications, the most important part is to have clean state handling, so that what the users see is always what they are interacting with.

\begin{figure}[H]
  \centering
  \includegraphics[width=0.6\textwidth]{../assets/current-refinements.png}
  \caption{Current Refinements in React InstantSearch\cite{ris-storybooks}}
  \label{figure:current-refinements}
\end{figure}

To avoid having users to keep the state of their search experience by themselves, the Algolia JavaScript helper~\cite{algolia-js-helper} was written. Its goal is to be a reference to the state of the search, and provide ways to change the state.

When an abstraction for the search state has been made, it becomes a lot more straight-forward to make reason about what a certain action actually does. For example, it becomes trivial to make for example a button that changes the page of the search results.

However, when using the JavaScript Helper, there are still two major roadblocks that are left in userspace. These roadblocks are linked to more advanced components, like for example a refinement list. A refinement list is a component that holds a list of possible filters which are relevant to the currently set refinements. It will also show how many results there would be, if that filter is chosen.

\begin{figure}[H]
  \centering
  \includegraphics[width=0.4\textwidth]{../assets/refinementlist.png}
  \caption{a RefinementList in React InstantSearch\cite{ris-storybooks}}
  \label{figure:refinementlist-ris}
\end{figure}

To make a component like a RefinementList, the struggle is one one side that a lot of different functions have to be called to be able to get the data, since behind the screens actually two API queries happen. One to get the counts and one to get the results. The second hurdle that would need to be crossed is the actual rendering. A user would need to decide which markup structure to use, as well as the accessibility and usability implications of it.

To solve both of these problems, in 2015 InstantSearch.js~\cite{instantsearch-js} was released by Algolia. It gives users of the library a hook for every widget to mount it to their application, as well as leaving the possibility to change some of the markup structure by having a templating system built-in.

InstantSearch.js uses React behind the scenes. React\cite{react-doc} is a popular JavaScript library for building user interfaces. One key thing of what React does is that it takes ownership of a part of a document structure. That -- ironically -- means that using InstantSearch.js in a React application is not a trivial operation.

Frameworks often impose a certain way of writing code that doesn't work well for libraries that are supposed to be used in a framework agnostic manner. For that reason, when making React InstantSearch~\cite{react-instantsearch} in 2016, the choice was made to not only expose components that mount to the DOM, but also expose another concept called connectors, which expose only the data that is relevant to a component, and a way to refine on it, relevant to that refinement.

This evolution makes for three distinct phases in how state has been handled, based on which library is used.

\subsection{Algolia JavaScript client} % (fold)
\label{sub:algolia_js_client}

When Algolia pivoted in 2013 to a SaaS product~\cite{algolia-blog-saas}, API clients~\cite{algolia-blog-lauch} for six programming languages were immediately available. JavaScript has always been important because of the focus on speed that Algolia has.

There are two major options to make a search experience. The first is to send a request from the frontend to the search service, where a JSON response is returned, which then is uses to render results in client side code. The other is sending that same request via a form on an application-specific backend, and constructing the results there, and eventually sending a completely new page back as a result to the user's query.

Algolia has a preference for clients to construct the results on the frontend, mostly because it's a lot faster. Algolia has infrastructure~\cite{algolia-infra} all around the world, and thus having an extra roundtrip to wherever the backend of a client is, can take hundreds of milliseconds more. Since search seems slow as soon as the timeout takes more than 50ms, every millisecond matters.

The JavaScript client~\cite{algolia-js-client} is the lowest useful level of abstraction, and handles things like retry strategies, dropped requests, dynamic routing based on location and request fallbacks. Because it originally was the only way to query Algolia, it also supports setting individual parameters before searching.

When using just the JavaScript client, it can be noticed that all state should be handled inside the application code. That means keeping track whether a certain refinement has been set and unset, and keeping this consistent within the application code. Furthermore, the application needs to be aware when Algolia should be queried, and then handling all of the incoming results, dealing with out-of-order responses etc.

Making a user interface with just the JavaScript client requires you to take ownership of the search state. An intuitive to deal with the state then, is saving it as part as the DOM, for example by setting attributes.

This however isn't very efficient, since that implies that when the query needs to be changed somehow, there need to happen a lot of checks to see if the extra refinement is possible.

\begin{figure}[H]
  \centering
  \includegraphics[width=0.6\textwidth, height=0.2\textheight, draft]{../assets/js-client-state.pdf}
  \caption{Handling state with the JavaScript client}
  \label{figure:js-client-state}
\end{figure}

Adding a ``Clear All'' button in this structure is not trivial. There needs to be either a query for all elements that need to be updated, or they need to be kept up to date in a list. Then for each of those elements the relevant attributes would need to be updated or removed.

% subsection algolia_js_client (end)

\subsection{Algolia JavaScript Helper} % (fold)
\label{sub:algolia_js_helper}

Before being a separate concept, the ``Helper'' was an integral part of the JavaScript API client~\cite{algolia-blog-js-client}~. In 2015 it was moved out as a standalone library with as its main feature handling state of a search experience.

This means that when making a search experience, a new instance --- further called \emph{the} Helper for that view --- of the Helper is created. When it is being initialized, it creates a bond with a JavaScript API client instance, to be used for sending the API calls to the servers.

This Helper exposes functions to modify the current state. It has three kinds of those functions, which are:

\begin{enumerate}
  \item facets
  \item filters
  \item query parameters
\end{enumerate}

The Helper uses an observer pattern to be able to express listening to what happens in the search state, regardless of where it comes from. By attaching listeners to the instance, it's possible to react to every change that happens, and update the UI. This way there's only one central source of truth.

By providing a central place of truth for search state, a lot of the complexity in user land disappears immediately. When making a component that needs to change a facet, but it is only active when a different filter has been set, becomes simple, since all that needs to be used is a check on {\tt helper.getState()}, and if the condition is met, a {\tt helper.toggleFacetRefinement()} can be called.

Every time a new search request needs to be fetched, the {\tt results} observer is being triggered, and the search user interface can be updated with those new results.

\begin{figure}[H]
  \centering
  \includegraphics[width=0.5\textwidth]{../assets/helper-cycle.pdf}
  \caption{Handling state with the JavaScript Helper\cite{js-helper-concepts}}
  \label{figure:js-helper-state}
\end{figure}

The functions of the Helper don't stop there, since more advanced use cases like ``Hierarchical faceting''\cite{hierarchical-faceting} are being handled. Hierarchical faceting is a concept, which requires the structure of the data to be formatted in a particular way to denote hierarchy. 

As an example for hierarchical facets, in figure \ref{figure:hierarchical-facets}. Here the {\tt hierarchicalCategory} attribute will be used as a hierarchical facet. To be able to do that, it should be an object, with {\tt lvlX} keys, where $X$ is the level of indentation. Inside each level $>$ is used as a level separator. That way an item could appear in multiple categories per level.

\begin{figure}[H]
  \centering
  \includegraphics[width=0.6\textwidth]{../assets/hierarchical-dashboard.png}
  \caption{Structure of a hierarchical facet in the Algolia dashboard}
  \label{figure:hierarchical-facets}
\end{figure}

% subsection algolia_js_Helper (end)

\subsection{InstantSearch.js} % (fold)
\label{sub:instantsearch_js}

rely on helper, expose it. Problem: very hard to make new components because you don't know how the Helper works, and there's only a carte blanche component. %%

\begin{figure}[H]
  \centering
  \includegraphics[width=0.6\textwidth, height=0.2\textheight, draft]{../assets/is-js-state.pdf}
  \caption{Handling state with InstantSearch.js v1}
  \label{figure:is-js-state}
\end{figure}

\subsubsection{Connectors} % (fold)
\label{ssub:instantsearch_js_connectors}

Connectors happened in v2 of instantsearch
-> expose each of the widgets cleanly
-> I helped in the design + implem %%

% subsubsection instantsearch_js_connectors (end)

% subsection instantsearch_js (end)

\subsection{React InstantSearch} % (fold)
\label{sub:react_instantearch}

React InstantSearch doesn't directly use the JavaScript Helper, but it rather stores its component state in a separate container. This container is being managed by an {\tt InstantSearchManager}. It will take in the parameters to change at any given moment by the components, and transfer that into both a state to be used by the components, as well as sending requests.

Similar to the newer v2 version of InstantSearch.js, described in \ref{ssub:instantsearch_js_connectors}, React InstantSearch introduced the concept of connectors and components, that together make a widget.

A connector will define to the InstantSearch Manager what parameters are being tracked, and thus will be updated on every time new results come in. These connectors expose a {\tt refine} function, which depending on the exact widget will be called with a string, array, object or number.

React InstantSearch doesn't stop at that place, it completely abstracts away the JavaScript Helper, and makes it impossible to actually use the Helper functions. This is done with a similar reason in how React abstracts away the DOM. When you'd change something in the JavaScript Helper, but not in the state tracking of React InstantSearch, it would cause confusion, and possibly collusions.

React InstantSearch proposes two different ways of countering this lack of visibility: the {\tt Configure} widget, and {\tt createConnector}.

\begin{figure}[H]
  \centering
  \includegraphics[width=0.6\textwidth, height=0.2\textheight, draft]{../assets/is-react-state.pdf}
  \caption{Handling state with React InstantSearch}
  \label{figure:is-react-state}
\end{figure}

\subsubsection{Configure}
\label{ssub:ris-configure}

The configure widget is a widget, which does not expose rendering, but rather only exists for the props being set on it. Every prop that is set on a Configure widget is being passed down towards the used Algolia JavaScript client.

\begin{lstlisting}[caption={Configure in an InstantSearch container},label={lst:ris-configure}]
const App = ({isDistinct, children}) => (
  <InstantSearch
    appId="latency"
    apiKey="6be0576ff61c053d5f9a3225e2a90f76"
    indexName="ikea"
  >
    <Configure distinct={isDistinct} />
    {children}
  </InstantSearch>
);
\end{lstlisting}

Here an app component is being made, which will take in as props a boolean which decides whether to set the {\tt DISTINCT} parameter to true or false, and the children of that app.

React causes a new render every time something changes in the props. Because of that behaviour, the mapping from the props on Configure, together with the props of other widgets can happen at render time.

\subsubsection{createConnector}
\label{ssub:ris-createconnector}

Every widget has its corresponding connector, as well as a few connectors that exist without default widget. A list of connectors that currently exist without default rendering implementation are:

\begin{itemize}
  \item {\tt connectAutoComplete}
  \item {\tt connectRange}
\end{itemize}

The AutoComplete and Range components only exist as a connector, because they imply significant style and DOM knowledge. Since there exist a few third-party implementations of these components that can be used with external data, it's fairly straightforward to use the connector Higher order Component as a way to pass through the data to the component, and call the {\tt refine()} function in click handlers. 

However providing a connector for every single possible widget would not be feasible. Therefore a factory function {\tt createConnector} exists.

TODO: explain createConnector

%%

% subsection react_instantearch (end)

\subsection{The future} % (fold)
\label{sub:the_future}

In the future, if InstantSearch Core is chosen as the new abstraction level, there would not be a need to abstract away the JavaScript Helper. 

Since InstantSearch Core, as described in section \ref{sec:organizing_state}, maps the state of the each component to its attribute name. In there it has a type and a value, which makes it very easy to know which widgets to render.

When using InstantSearch Core, the libraries using it would be able to expose it as a way to expand towards other connectors and widgets that aren't provided by default.

That way, the experience of creating other connectors will feel like a native feature, instead of a separate library that a user would need to learn and understand first.

\begin{figure}[H]
  \centering
  \includegraphics[width=0.6\textwidth, height=0.2\textheight, draft]{../assets/is-core-state.pdf}
  \caption{Handling state with the InstantSearch Core}
  \label{figure:is-core-state}
\end{figure}

% subsection the_future (end)

% section making_search_interfaces_with_algolia (end)
