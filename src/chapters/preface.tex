% !TEX root = ../main.tex

\chapter{Preface}% (fold)
\label{chp:preface}

The goal of a library is a noble one. It's a way to share knowledge someone has, in a way that is available to everyone. The users of the library don't need to be able to know how the library is written to be able to use it. But it goes further than that, because in the other direction it also is very important that libraries don't constrict what you can do with it.

Finding a balance between abstracting away the difficult parts, just exposing both simple ways to tweak things, but also exposing an API that allows \emph{anyone} that wants to dig further to get into the nitty-gritty of the library and find reasons why something is working like it is, and changing it if needed.

The interesting things about writing libraries compared to writing applications is that not the users are the people that make full use of the library, but the users' users. This gives a deep going relationship between someone who depends on a library and its maintainer.

Almost all recent innovations in front-end programming have been explored in open source solutions, which means a lot can be learned, just by looking at how other people solve a certain solution, so a mix of different flows can be created, without coming up with each of the ideas. As Isaac Newton attributed Bernard of Chartres a lot more nicely than I could: 

\begin{quotation}
  If I have seen further, it is by standing on the shoulders of giants
\end{quotation}

I strongly believe that everything we do, how small or big it is has use in being shared with everyone. 
