% !TEX root = ../main.tex

\chapter{Preface}% (fold)
\label{chp:preface}

The goal of a library is a noble one. It's a way to share knowledge someone has, in a way that is available to everyone. This goes up for libraries that hold books, and have been an important source of sharing knowledge over the years, but might even be more true for programming libraries. 

By encapsulating knowledge in a library, not everyone needs to know how to make niche know, or tedious operations. But it goes further than that, because in the other direction it also is very important that libraries don't constrict what you can do with it.

The challenge when building libraries that will be used in other libraries is finding a balance between abstracting away the difficult parts, just exposing a simple way to tweak things, but also exposing an API that allows \emph{anyone} that wants to dig further to get into the nitty-gritty of the library and find reasons why something is working like it is, and changing it if needed.

The interesting aspect of writing libraries compared to writing applications is that not the users are the people that make full use of the library, but the users' users. This gives a deep relationship between someone who depends on a library and its maintainer.

Almost all recent innovations in front-end and web programming have been explored in open source solutions, which means a lot can be learned, just by looking at how other people solve a certain solution, so a mix of different flows can be created, without coming up with each of the ideas. As Isaac Newton attributed\cite{newton-giants} Bernard of Chartres\cite{quote-giants-source} a lot more nicely than I could: ``If I have seen further, it is by standing on the shoulders of giants''. Working together as a community of developers and engineers, the possibilities are nearly endless.

I strongly believe that everything we do, how small or big it is has use in being shared with everyone.
