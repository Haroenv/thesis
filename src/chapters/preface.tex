% !TEX root = ../main.tex

\chapter{Preface}% (fold)
\label{chp:preface}

The goal of a \gls{library} is a noble one. It's a way to share knowledge someone has, in a way that is available to everyone. This goes up for libraries that hold books, and have been an important source of sharing knowledge over the ages, but might even be more true for programming libraries. 

By encapsulating knowledge in a \gls{library}, not everyone needs to know niche knowledge or how to execute tedious operations. But it goes further than that, because in the other direction it also is very important that libraries don't constrict the user in their intentions.

The challenge when building libraries that will be used in other libraries is finding a balance between abstracting away the difficult parts, just exposing a simple way to tweak things, but also exposing an \acrshort{api} that allows \emph{anyone} that wants to dig further to get into the nitty-gritty of the \gls{library} and find reasons why something is working like it is, and changing it if needed.

When writing a \gls{library}, it's not just written for the developer that will consume the library, it's also written for the users of that developer. This gives a deep relationship between someone who depends on a \gls{library} and its maintainer.

Almost all recent innovations in front-end and web programming have been explored in open source solutions. That means a lot can be learned, just by looking at how other people solve a certain problem. A mix of different approaches can be created, without coming up with each of the ideas from scratch. 

As Isaac Newton\cite{newton-giants} attributed Bernard of Chartres\cite{quote-giants-source} a lot better than I could: ``If I have seen further, it is by standing on the shoulders of giants''. Working together as a community of developers and engineers, the possibilities are nearly endless. I strongly believe that everything we do --- regardless how small or big it is --- has use in being shared with everyone.
